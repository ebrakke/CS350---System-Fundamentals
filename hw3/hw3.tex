\documentclass[11pt]{article}
\usepackage{amssymb,amsmath,amsthm,graphicx}
\usepackage{fancyhdr}

\def\shownotes{1}   % set 1 for version with author notes
                    % set 0 for no notes



%uncomment to get hyperlinks
%\usepackage{hyperref}

%%%%%%%%%%%%%%%%%%%%%%%%%%%%%%%%%%%%%%%%%%%%%%%%%%%%%%%%%%%%%%
%Some macros (you can ignore everything until "end of macros")

\topmargin 0pt \advance \topmargin by -\headheight \advance
\topmargin by -\headsep

\textheight 8.9in

\oddsidemargin 0pt \evensidemargin \oddsidemargin \marginparwidth
0.5in

\textwidth 6.5in

%%%%%%

\providecommand{\vs}{vs. }
\providecommand{\ie}{\emph{i.e.,} }
\providecommand{\eg}{\emph{e.g.,} }
\providecommand{\cf}{\emph{cf.,} }
\providecommand{\etc}{\emph{etc.} }

\newcommand{\getsr}{\gets_{\mbox{\tiny R}}}
\newcommand{\bits}{\{0,1\}}
\newcommand{\bit}{\{0,1\}}
\newcommand{\Ex}{\mathbb{E}}
\newcommand{\eqdef}{\stackrel{def}{=}}
\newcommand{\To}{\rightarrow}
\newcommand{\e}{\epsilon}
\newcommand{\R}{\mathbb{R}}
\newcommand{\N}{\mathbb{N}}
\newcommand{\Gen}{\mathsf{Gen}}
\newcommand{\Enc}{\mathsf{Enc}}
\newcommand{\Dec}{\mathsf{Dec}}
\newcommand{\Sign}{\mathsf{Sign}}
\newcommand{\Ver}{\mathsf{Ver}}

\providecommand{\mypara}[1]{\smallskip\noindent\emph{#1} }
\providecommand{\myparab}[1]{\smallskip\noindent\textbf{#1} }
\providecommand{\myparasc}[1]{\smallskip\noindent\textsc{#1} }
\providecommand{\para}{\smallskip\noindent}


\newtheorem{theorem}{Theorem}
\theoremstyle{definition}
\newtheorem{ex}{Exercise}
\newtheorem{definition}{Definition}

%%%%%%%  Author Notes %%%%%%%d
%
\ifnum\shownotes=1
\newcommand{\authnote}[2]{{ $\ll$\textsf{\footnotesize #1 notes: #2}$\gg$}}
\else
\newcommand{\authnote}[2]{}
\fi
\newcommand{\Snote}[1]{{\authnote{Solution}{#1}}}
\newcommand{\Inote}[1]{{\authnote{Solution}{#1}}}
\newcommand{\Ichanged}[1]{{\authnote{Changed}{#1}}}
%%%%%%%%%%%%%%%%%%%%%%%%%%%%%%%%%

\newcommand{\VAR}{\mathrm{VAR}}



% end of macros
%%%%%%%%%%%%%%%%%%%%%%%%%%%%%%%%%%%%%%%%%%%%%%%%%%%%%%%%%%%%%%


% page counting, header/footer
\usepackage{fancyhdr}
\pagestyle{fancy}
\lhead{\footnotesize \parbox{11cm}{CS350, Boston University, Fall 2015} }
\rhead{Erik Brakke}
\renewcommand{\headheight}{24pt}

\begin{document}

\title{Homework 3}
\author{Erik Brakke}
\maketitle

\thispagestyle{fancy}
 
 
\section*{Answer 1}
\begin{enumerate}
	\item[a.] Using the equation, $E = Z_{\alpha/2}\frac{S}{\sqrt{n}}$, we can plug in what we know\\
	$.4 = 2.06*S / \sqrt{40}$\\
	This gives us a standard dev of 1.23
	
	\item[b.] $E = Z_{.01/2}\frac{S}{\sqrt{n}}$\\
	$E = 2.58 * 1.23 / \sqrt{40} = .5$\\
	So the new interval would be $3 \pm .5$\\
	
	\item[c.] $E = Z_{\alpha/2}\frac{S}{\sqrt{n}}$\\
	$.1 = 2.06*1.23 / \sqrt{n}$\\
	$n = 641.01$ so you would need to sample 642 people
\end{enumerate}

\section*{Answer 2}
Submitting late (Note this question is the only one submitted late, Question 1,3, and 4 were submitted on time)
See the code in the 'src' folder

\begin{enumerate}
	\item[a.] Log file is written to q2a\_log.csv\\
	The observed average using Excel was $w = 11.25, q = 12.17, T_q = .142, T_w = .127, T_s = .01495$\\
	Because there were so many trials, the errors computed are neglegible (in most cases less than .001)\\
	The computed values are $q = \rho / (1 - \rho) = 9 / .1 = 9, w = 8.1, T_q = .15, T_w = .135, T_s = .015$\\
	$T_s$ was the closest value.  Perhaps if more runs were sampled, or the simulation ran longer, the values would be closer
	
	\item[b.] Computed values: $q = .75/.25 = 3, w = 2.25, T_q = .06, T_w = .0375, T_s = .015$\\
			Actual: $q = 3.92, w = 3.14, T_q = .072, T_w = .057, T_s = .0153$\\
			STD on all are very small, so error is very small (sample size 9900)\\
			These values are close, though not exactly what is computed
	
	\item[c.] Computed values: $q = .975/.025 = 39, w = 2.25, T_q = .06, T_w = .0375, T_s = .015$\\
	Actual: $q = 27.27, w = 26.21, T_q = .275, T_w = .26, T_s = .015$\\
	Error:  $E_q = 2.58 * (9.25 / \sqrt{6356}) = \pm.11$ Errors are very small at 99\% confidence\\
	If the system was run for longer, the number might converge as multiple runs of this yeild very different averages (some close some not)
	
	\item[d.] This will go to infinity becuase $\rho > 1$\\
	Looking at the log, the average q is huge (1049), and the standard deviation is 251.  This system cannot be properly analyzed because it will grow to infinity
\end{enumerate}

\section*{Answer 3}
Converted everything from seconds to milliseconds, $\lambda = 20$, capcity $= 24000B/sec$
\begin{enumerate}
	\item[a.] We should use M/G/1 because the arrival rate is Poisson and we know the know the normalized value of service time standard deviation. Because the packet
	size is unform, and the packet size is what determines the service time, we know what the mean service time is and the standard deviation.
	
	\item[b.] $T_s = ((100 + 1500) / 2) / 24000 = .033 ms$
	\item[c.] $STD = ((1500 - 100) / 24000) / \sqrt{12} = .017ms$\\
	\item[d.] $A = \frac{1}{2}(1 + (\frac{.017}{.033})^2) = .63$\\
	$\rho = 20 * .033 = .66$\\
	$q = \frac{.66^2 * .63}{.34} + .66 = 1.47$
	
	\item[e.] $w = q - \rho = .81$
	\item[f.] $T_q = q / \lambda = 1.47 / 20 = .0735ms$
	\item[g.] $\Pr[S_0] = 1 - \rho = .33$
\end{enumerate}

\section*{Answer 4}
\begin{enumerate}
	\item[a.] \includegraphics*[scale=0.5]{q4a.jpg}
	\item[b.] Arrival rate to CPU system, $X = 40 + (0.1*X) + (0.5 * 0.1 * X + 0.4*X) = 80$\\
	Arrival rate to Disk, $Y = 0.1*X = 8$\\
	Arrival rate to Network, $Z = 0.4*X + 0.5*Y = 36$\\
	\newline
	First, find $T_q$ of CPU (M/M/2)\\
	$\rho = \lambda * T_s / N = 80 * .02 / 2 = .8$\\
	$C = \frac{2\rho^2}{1 + \rho} = .71$\\
	$q = C * \frac{\rho}{1 - \rho} + N*\rho = .71 * \frac{.8}{.2} + 1.6 = 4.44$\\
	\newline
	Next, $T_q$ of the Disk (M/M/1)\\
	$\rho = \lambda * T_s = 8 * .1 = .8$\\
	$q = \frac{\rho}{1 - \rho} = .8 / .2 = 4$\\
	\newline
	Next, $T_q of the network (M/M/1)$\\
	$\rho = \lambda * T_s = 36 * .025 = .9$\\
	$q = \frac{.9}{.1} = 9$\\
	\newline
	Total $q = 9 + 4 + 4.44 = 17.44$
	Total $T_q = 17.44 / 80 = .218$
	
	\item[c.] The bottleneck is the the network because it has the highest utilization and highest $q$\\
	\item[d.] We need to find which processes will have $\rho > 1$ at a given arrival rate\\
	For cpu, $\lambda = 1 / .01 = 100$\\
	disk, $\lambda = 1 / .1 = 10$\\
	network $\lambda = 1 / .025 = 40$\\
	For the arrival rate $X$ to reach 100 at the CPU, $\lambda$ to the whole system would have to be $100 - .55(100) = 45$\\
	For the arrival rate $Y$ to reach 10 at the Disk, $10 = .1(X), X = 100, \lambda = 45$\\
	For the arrival rate $Z$ to reach 40 at the Network, $40 = .45X, X = 88.88, \lambda = 88.88 - .55(88.88)= 40$\\
	\newline
	We can see that if the arrival rate to the system is over 40, then the queue in the Network queue will grow infinitely long
	
\end{enumerate}
\end{document} 
