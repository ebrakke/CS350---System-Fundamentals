\documentclass[11pt]{article}
\usepackage{amssymb,amsmath,amsthm,graphicx}
\usepackage{fancyhdr}

\def\shownotes{1}   % set 1 for version with author notes
                    % set 0 for no notes



%uncomment to get hyperlinks
%\usepackage{hyperref}

%%%%%%%%%%%%%%%%%%%%%%%%%%%%%%%%%%%%%%%%%%%%%%%%%%%%%%%%%%%%%%
%Some macros (you can ignore everything until "end of macros")

\topmargin 0pt \advance \topmargin by -\headheight \advance
\topmargin by -\headsep

\textheight 8.9in

\oddsidemargin 0pt \evensidemargin \oddsidemargin \marginparwidth
0.5in

\textwidth 6.5in

%%%%%%

\providecommand{\vs}{vs. }
\providecommand{\ie}{\emph{i.e.,} }
\providecommand{\eg}{\emph{e.g.,} }
\providecommand{\cf}{\emph{cf.,} }
\providecommand{\etc}{\emph{etc.} }

\newcommand{\getsr}{\gets_{\mbox{\tiny R}}}
\newcommand{\bits}{\{0,1\}}
\newcommand{\bit}{\{0,1\}}
\newcommand{\Ex}{\mathbb{E}}
\newcommand{\eqdef}{\stackrel{def}{=}}
\newcommand{\To}{\rightarrow}
\newcommand{\e}{\epsilon}
\newcommand{\R}{\mathbb{R}}
\newcommand{\N}{\mathbb{N}}
\newcommand{\Gen}{\mathsf{Gen}}
\newcommand{\Enc}{\mathsf{Enc}}
\newcommand{\Dec}{\mathsf{Dec}}
\newcommand{\Sign}{\mathsf{Sign}}
\newcommand{\Ver}{\mathsf{Ver}}

\providecommand{\mypara}[1]{\smallskip\noindent\emph{#1} }
\providecommand{\myparab}[1]{\smallskip\noindent\textbf{#1} }
\providecommand{\myparasc}[1]{\smallskip\noindent\textsc{#1} }
\providecommand{\para}{\smallskip\noindent}


\newtheorem{theorem}{Theorem}
\theoremstyle{definition}
\newtheorem{ex}{Exercise}
\newtheorem{definition}{Definition}

%%%%%%%  Author Notes %%%%%%%d
%
\ifnum\shownotes=1
\newcommand{\authnote}[2]{{ $\ll$\textsf{\footnotesize #1 notes: #2}$\gg$}}
\else
\newcommand{\authnote}[2]{}
\fi
\newcommand{\Snote}[1]{{\authnote{Solution}{#1}}}
\newcommand{\Inote}[1]{{\authnote{Solution}{#1}}}
\newcommand{\Ichanged}[1]{{\authnote{Changed}{#1}}}
%%%%%%%%%%%%%%%%%%%%%%%%%%%%%%%%%

\newcommand{\VAR}{\mathrm{VAR}}



% end of macros
%%%%%%%%%%%%%%%%%%%%%%%%%%%%%%%%%%%%%%%%%%%%%%%%%%%%%%%%%%%%%%


% page counting, header/footer
\usepackage{fancyhdr}
\pagestyle{fancy}
\lhead{\footnotesize \parbox{11cm}{CS350, Boston University, Fall 2015} }
\rhead{Erik Brakke}
\renewcommand{\headheight}{24pt}

\begin{document}

\title{Homework Number}
\author{Erik Brakke}
\maketitle

\thispagestyle{fancy} 
 
\section*{Answer 1}
\begin{enumerate}
	\item[a.]  CPU utilization: $\frac{5ms \text{Busy Time}}{19ms \text{Total time}} = .263$ 26.3\%\\
	Disk utilization: $\frac{8ms}{19ms} = .421$ 42.1\%\\
	Network utilization: $\frac{6ms}{19ms} = .316$ 31.6\%

	\item[b.] $1 req / .019s = 52.63req / second$

	\item[c.] Total time: 28ms (Assuming first process wins in ties)\\
	CPU utilization: $\frac{12ms}{28ms} = .429$ 42.9\%\\
	Disk utilization: $\frac{21ms}{28ms} = .750$ 75.0\%\\
	Network utilization: $\frac{12ms}{28ms} = .429$ 42.9\%\\

	\item[d.] $2.26 req / .028s = 80.71req/second$ The .26 came from the 7 seconds of a new process that is handled on the first thread

	\item[e.] Total time: 35ms (Same assumption as above)\\
	CPU utilization: $\frac{20ms}{35ms} = .571$ 57.1\%\\
	Disk utilization: $\frac{32ms}{35ms} = .914$ 91.4\%\\
	Network utilization: $\frac{18ms}{35ms} = .514$ 51.4\%\\
	Capacity: $3.63 req / .035s = 103.71req/second$

	\item[f.] The max capacity is 110 requests per second.  (4 threads, and 4.84 requests being processed)
	\item[g.] The bottleneck is the disk
\end{enumerate}

\section*{Answer 2}
\begin{enumerate}
	\item[a.] MPL = 4, CPU utilization = 26/44, $speedup = \frac{1}{1-\frac{26}{44} + \frac{26/44}{r}}$ 
	\item[b.] Take $r \rarrow \inf$, $\frac{1}{1 - (26/44)} = 2.44$
	\item[c.] Disk util = 42/44, Network util = 24/44, $speedup_{disk} = \frac{1}{1 - \frac{42}{44}} = 22$, $speedup_{network} = \frac{1}{1 - \frac{24}{44}} = 2.2$
	\item[d.] None
\end{enumerate}

\section*{Answer 3}
\begin{enumerate}
	\item[a.] cache util = .95\\
	Scenario 1 $speedup = \frac{1}{1 - .05 + .05 / 1.5} = 1.01$\\
	Scenario 2 $speedup = \frac{1}{1 - .95 + .95 / 1.2} = 1.19$\\

	\item[b.] We want to find the intersection of the two speedups\\
	$\frac{1}{1 - h + h/1.2} = \frac{1}{1 - (1-h) + (1-h)/1.5}$ This happens when $h = .667$\\
	So when hit rate < 66.7\%, take the memory speedup, else take the cache speedup

	\item[c.] $speedup = \frac{1}{(h/1.2) + (h/1.5)}$
\end{enumerate}

\section*{Answer 4}
\begin{enumerate}
	\item[a.] $150/15000 = .1$ wasted for overhead  
	$T(H,P) = 9 * (1 - P)^H$

	\item[b.] $T(1,.01) = 9 * (.99) = 8.91Mpbs$
	\item[c.] $6 < 9 * (1 - P)^H$\\
	$.66 < (1 - P)^H$\\
	$\sqrt[n]{.66} < 1 - P$\\
	$P < 1 - \sqrt[n]{.66}$]]
\end{enumerate}

\end{document} 