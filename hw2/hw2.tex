\documentclass[11pt]{article}
\usepackage{amssymb,amsmath,amsthm,graphicx}
\usepackage{fancyhdr}

\def\shownotes{1}   % set 1 for version with author notes
                    % set 0 for no notes



%uncomment to get hyperlinks
%\usepackage{hyperref}

%%%%%%%%%%%%%%%%%%%%%%%%%%%%%%%%%%%%%%%%%%%%%%%%%%%%%%%%%%%%%%
%Some macros (you can ignore everything until "end of macros")

\topmargin 0pt \advance \topmargin by -\headheight \advance
\topmargin by -\headsep

\textheight 8.9in

\oddsidemargin 0pt \evensidemargin \oddsidemargin \marginparwidth
0.5in

\textwidth 6.5in

%%%%%%

\providecommand{\vs}{vs. }
\providecommand{\ie}{\emph{i.e.,} }
\providecommand{\eg}{\emph{e.g.,} }
\providecommand{\cf}{\emph{cf.,} }
\providecommand{\etc}{\emph{etc.} }

\newcommand{\getsr}{\gets_{\mbox{\tiny R}}}
\newcommand{\bits}{\{0,1\}}
\newcommand{\bit}{\{0,1\}}
\newcommand{\Ex}{\mathbb{E}}
\newcommand{\eqdef}{\stackrel{def}{=}}
\newcommand{\To}{\rightarrow}
\newcommand{\e}{\epsilon}
\newcommand{\R}{\mathbb{R}}
\newcommand{\N}{\mathbb{N}}
\newcommand{\Gen}{\mathsf{Gen}}
\newcommand{\Enc}{\mathsf{Enc}}
\newcommand{\Dec}{\mathsf{Dec}}
\newcommand{\Sign}{\mathsf{Sign}}
\newcommand{\Ver}{\mathsf{Ver}}

\providecommand{\mypara}[1]{\smallskip\noindent\emph{#1} }
\providecommand{\myparab}[1]{\smallskip\noindent\textbf{#1} }
\providecommand{\myparasc}[1]{\smallskip\noindent\textsc{#1} }
\providecommand{\para}{\smallskip\noindent}


\newtheorem{theorem}{Theorem}
\theoremstyle{definition}
\newtheorem{ex}{Exercise}
\newtheorem{definition}{Definition}

%%%%%%%  Author Notes %%%%%%%d
%
\ifnum\shownotes=1
\newcommand{\authnote}[2]{{ $\ll$\textsf{\footnotesize #1 notes: #2}$\gg$}}
\else
\newcommand{\authnote}[2]{}
\fi
\newcommand{\Snote}[1]{{\authnote{Solution}{#1}}}
\newcommand{\Inote}[1]{{\authnote{Solution}{#1}}}
\newcommand{\Ichanged}[1]{{\authnote{Changed}{#1}}}
%%%%%%%%%%%%%%%%%%%%%%%%%%%%%%%%%

\newcommand{\VAR}{\mathrm{VAR}}



% end of macros
%%%%%%%%%%%%%%%%%%%%%%%%%%%%%%%%%%%%%%%%%%%%%%%%%%%%%%%%%%%%%%


% page counting, header/footer
\usepackage{fancyhdr}
\pagestyle{fancy}
\lhead{\footnotesize \parbox{11cm}{CS350, Boston University, Fall 2015} }
\rhead{Erik Brakke}
\renewcommand{\headheight}{24pt}

\begin{document}

\title{Homework 2}
\author{Erik Brakke}
\maketitle

\thispagestyle{fancy}
 
 
\section*{Answer 1}
\begin{enumerate}
	\item[a.] $\Pr[\text{no connections refuesed}] = \Pr[\text{server available for each connection}] = .96^{20} = .442$
	\item[b.] $\Pr[\text{exactly 1 connection refused}] = 20 * .04 * .96^{19} = .368$
	\item[c.] $\Pr[\text{3 or less will be refused}] = \Pr[\text{None refused}] + \Pr[\text{1 refused}] + \Pr[\text{2 refused}] + \Pr[\text{3 refused}] = .442 + .368 + .146 + .036 = .992
	\item[d.] $\Pr[\text{4 or more consecutive}] = .94 ^ 4 = .849$
	\item[e.] Expected value of successful request before failure = $p / 1-p = .96 / .04 = 24$
	\item[f.] Expected successful requests in 50 = $np = 50 * .96 = 48$
	\item[g.] 
	\item[h.] 
\end{enumerate}

\section*{Answer 2}
\begin{enumerate}
	\item[a.] $f(x) = \frac{2^x}{x!}e^{-2}$
	\item[b.] $f(x > 3) = 1 - (f(x = 3) + f(x=2) + f(x=1)) = 1 - (.18 + .27 + .27) = .28$
	\item[c.] $f(x) = 1 - e^{-.2x}$
	\item[d.] $1 - f(8) = 1 - .20 = .80$
	\item[e.] $std = 1/\lambda = 1 / 200 = 5 ms$
	\newline
	\item[a.] capacity = $1 / .0048 = 208.33 req/sec$
	\item[b.] 
	\item[c.] $\rho = \lambda * T_s = 200 * .0048 = .96$\\
	$w = \rho^2/(1-\rho) = 23.04$ req waiting in the queue
	\item[d.] $T_w = w/\lambda = 23.04 / 200 = 115ms$
	\item[e.] $Slowdown = 1/(1-\rho) = 1/.04 = 25$ 
\end{enumerate}

\section*{Answer 3}
\begin{enumerate}
	\item[a.] \includegraphics*[scale=0.5]{Q3A.png}
	Yes they do match.  We generated 100 exponential variables, so it would makes sense that these random variable should follow the CDF of exponential random variable.  Obviously the graph I generated
	was a bit more jagged than the ideal graph, but given a sufficient amount of points, the graph would continue to smooth outer
	
	\item[b.] $f(x) = \frac{\lambda^x}{x!}e^{-\lambda}\\
	\includegraphics*[scale=0.5]{q3b.png}
\end{enumerate}

\section*{Answer 4}
\begin{enumerate}
	\item[a.] $f(x) = \{1: .5, 3: .3, 10: .14, 30: .06, else: 0\}$
	\item[b.] $E[x] = 1*.5 + 3*.3 + 10*.14 + 30*.06 = 4.6seconds$
	\item[c.] $var = \sqrt{.5*(1 - 4.6)^2 + .3*(3-4.6)^2 + .14*(10-4.6)^2 + .06*(30 - 4.6)^2} = 7.07
	\item[d.] See code
\end{enumerate}

\section*{Answer 5}
\begin{enumerate}
	\item[a.] \includegraphics*[scale=0.5]{q5a.png}\\
	The plots match roughly.  $\Pr[X < 0] = .42$ when it should really be = .50, though as we look at x<1, x<2, x<3, etc... we see that the CDF follows the actual normal CDF more closely.
	This is because we only took 100 samples.  If we took more we would see the distribution following the actual more closely
	
	\item[b.]
\end{document} 