\documentclass[11pt]{article}
\usepackage{amssymb,amsmath,amsthm,graphicx}
\usepackage{fancyhdr}

\def\shownotes{1}   % set 1 for version with author notes
                    % set 0 for no notes



%uncomment to get hyperlinks
%\usepackage{hyperref}

%%%%%%%%%%%%%%%%%%%%%%%%%%%%%%%%%%%%%%%%%%%%%%%%%%%%%%%%%%%%%%
%Some macros (you can ignore everything until "end of macros")

\topmargin 0pt \advance \topmargin by -\headheight \advance
\topmargin by -\headsep

\textheight 8.9in

\oddsidemargin 0pt \evensidemargin \oddsidemargin \marginparwidth
0.5in

\textwidth 6.5in

%%%%%%

\providecommand{\vs}{vs. }
\providecommand{\ie}{\emph{i.e.,} }
\providecommand{\eg}{\emph{e.g.,} }
\providecommand{\cf}{\emph{cf.,} }
\providecommand{\etc}{\emph{etc.} }

\newcommand{\getsr}{\gets_{\mbox{\tiny R}}}
\newcommand{\bits}{\{0,1\}}
\newcommand{\bit}{\{0,1\}}
\newcommand{\Ex}{\mathbb{E}}
\newcommand{\eqdef}{\stackrel{def}{=}}
\newcommand{\To}{\rightarrow}
\newcommand{\e}{\epsilon}
\newcommand{\R}{\mathbb{R}}
\newcommand{\N}{\mathbb{N}}
\newcommand{\Gen}{\mathsf{Gen}}
\newcommand{\Enc}{\mathsf{Enc}}
\newcommand{\Dec}{\mathsf{Dec}}
\newcommand{\Sign}{\mathsf{Sign}}
\newcommand{\Ver}{\mathsf{Ver}}

\providecommand{\mypara}[1]{\smallskip\noindent\emph{#1} }
\providecommand{\myparab}[1]{\smallskip\noindent\textbf{#1} }
\providecommand{\myparasc}[1]{\smallskip\noindent\textsc{#1} }
\providecommand{\para}{\smallskip\noindent}


\newtheorem{theorem}{Theorem}
\theoremstyle{definition}
\newtheorem{ex}{Exercise}
\newtheorem{definition}{Definition}

%%%%%%%  Author Notes %%%%%%%d
%
\ifnum\shownotes=1
\newcommand{\authnote}[2]{{ $\ll$\textsf{\footnotesize #1 notes: #2}$\gg$}}
\else
\newcommand{\authnote}[2]{}
\fi
\newcommand{\Snote}[1]{{\authnote{Solution}{#1}}}
\newcommand{\Inote}[1]{{\authnote{Solution}{#1}}}
\newcommand{\Ichanged}[1]{{\authnote{Changed}{#1}}}
%%%%%%%%%%%%%%%%%%%%%%%%%%%%%%%%%

\newcommand{\VAR}{\mathrm{VAR}}



% end of macros
%%%%%%%%%%%%%%%%%%%%%%%%%%%%%%%%%%%%%%%%%%%%%%%%%%%%%%%%%%%%%%


% page counting, header/footer
\usepackage{fancyhdr}
\pagestyle{fancy}
\lhead{\footnotesize \parbox{11cm}{CS350, Boston University, Fall 2015} }
\rhead{Erik Brakke}
\renewcommand{\headheight}{24pt}

\begin{document}

\title{Homework 4}
\author{Erik Brakke}
\maketitle

\thispagestyle{fancy}
 
 
\section*{Answer 1}
Converted everything to milliseconds
\begin{enumerate}
	\item[a.] MM1K makes sense here because here because arrivals are poisson, service time is exponential, there is a single server, and there is a finite buffer
	
	\item[b.] $T_s = 20, \lambda = .05, \rho = 1$, so $q = K/2 = 1.5$\\
		  $\Pr(S_k) = 1/K+1 = .25$, so $\lambda' = .05*(.75) = .0375$ making $\rho_{effective} = .75$\\
		  Thus, $w = 1.5 - .75 = .75$
	\item[c.] $\Pr(S_k) = .25$, so to find the expected value of a blips, we want $2000 * .25 = 500$\\
		To find plops, we want to find $\Pr(S_0) = 1 - \rho_{effective} = .25$\\
		Thus, we expect 500 plops
	\item[d.] $T_q = q / \lambda' = 1.5 / .0375 = 40$
	\item[e.] $q = 5, \Pr(S_k) = 1/11, \lambda' = (.05)*(10/11) = .045$\\
		  $T_q = 5 / .045 = 111.11$
	\item[f.] This makes sense.  Increasing K will just increase the mean time that a packet is in the system.  The system cannot process any faster, but it is just holding more packets at a time in the buffer
	\item[g.] The advantage of increased buffer sizes is that less packets will be dropped and therefore there will be less retransmission of packets.  However, as the buffers increase, more space is taken up in the server
		  and the packets will just sit in the buffer longer.  If packets are time sensitive, then Increasing buffer size will not make them get to  the destination faster
	\item[h.] I would recommend a small value for K.  Loss of a few packets is not detremental to the experience of watching a movie (i.e. loss of quality is preferred).  Having a large K would just lead to packets taking
		  longer to reach the destination, resulting in video that pauses frequently as it waits for the next packet.  My answer is no different for Teleconferencing as it has the same properties.  Loss of quality is
		  preferred over a long delay to hear the one's voice
\end{enumerate}

\section*{Answer 2}
\begin{enumerate}
	\item[a.] Expected results: $\rho = 1, \Pr(S_k) = .25, q = 1.5, T_q = 1.5/37.5 = 40ms$\\
	Using Excel, Actual results were $\Pr(S_k) = .2475, q = 1.54 \pm .0012, T_q = 40.9ms \pm .005$\\
	$E_q = 1.96 * .05 / \sqrt{6058} = .0012$\\
	$E_{Tq} = 1.96 * .0002 / \sqrt{6058} = .005ms$\\
	The results were very close to actual in this case\\
	
	\item[b.] Expected results: $\rho = .75, \Pr(S_k) = \frac{.25 * .75^3}{1 - .75^4} = .154, \lambda' = 42.3, q = .63, T_q = .014$\\
	Actual results were $\Pr(S_k) = .136, q = 1.04 \pm .0001, T_q = 26ms \pm .002$\\
	$E_q = 1.96 * .058 / \sqrt{5881} = .0014$\\
	$E_{Tq} = 1.96 * .0001 / \sqrt{5881} = .002ms$\\
	These results were farther off from the expected, but a much longer simulation may have yeilded closer results
	
	\item[c.] Expected results: $\rho = .975, \Pr(S_k) = \frac{.025 * .975^3}{1 -.975^4} = .24, \lambda' = 49.4, q = .741, T_q = .015$\\
	Actual results were $\Pr(S_k) = .245, q = 1.54 \pm .0011, T_q = 30ms \pm .005$\\
	$E_q = 1.96 * .057 / \sqrt{10342} = .0011$\\
	$E_{Tq} = 1.96 * .0003 / \sqrt{10342} = .005ms$\\
	The dropped percentage was almost spot on, however $q$ and $t_q$ were almost double what they should have been
	
	\item[d.] Results: $q = 1.09\pm .00013, T_q = 24ms \pm .02, Pr(S_k) = .085$
	$E_q = 1.96 * .006 / \sqrt{8002} = .00013$\\
	$E_{Tq} = 1.96 * .001 / \sqrt{8002} = .0219ms$\\
	The rejection probability is much less.  This is because we know that service time will always be .015, then requests will not have to wait as long and therefore	
	there will be more buffer space available for incoming requests
	
	\item[e.] Results: $q = 1.52, T_q = 27ms, Pr(S_k) = .183$
	Again, $q$ and $T_q$ were about the same, but the probability of a request being dropped is much less.  This expected when the service time is deterministic

\end{enumerate}
	
\end{document} 
